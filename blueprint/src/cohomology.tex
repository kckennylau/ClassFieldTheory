% In this file you should put the actual content of the blueprint.
% It will be used both by the web and the print version.
% It should *not* include the \begin{document}
%
% If you want to split the blueprint content into several files then
% the current file can be a simple sequence of \input. Otherwise It
% can start with a \section or \chapter for instance.


\chapter{Group Cohomology}

\section{Generalities}

Let $G$ be a group.
By a $G$-module, we shall mean a module $M$ over the integral group ring $\Z[G]$, on which we
write scalars on the left. This is equivalent to giving a monoid homomorphism $G \to End_\Z M$.
The category of $G$-modules is in Mathlib as \texttt{Rep Int G}.

Given a $G$-module $M$, there is a cochain complex $C^n(G,M)$ of $\Z$-modules
indexed by $n \in \N$, which consists of the functions $G^n \to M$ with some appropriately defined
differentials.
The cochain complex $C^n(G,M)$ is functorial in $M$ and is defined in Mathlib as
\texttt{groupCohomology.cochainsFunctor}.
The cohomology groups of this chain complex are called the cohomology groups of the
$G$-module $M$, and are written $H^n(G,M)$. These are defined in Mathlib as
\texttt{groupCohomology M n}.

The next two lemmas are a little technical.

\begin{lemma}
  \label{lem:groupCohomology.exact1}
  \lean{groupCohomology.exact1}
  The functor taking $M$ to $C^\bullet (G,M)$ preserves finite limits.
\end{lemma}

\begin{lemma}
  \label{lem:groupCohomology.exact2}
  \lean{groupCohomology.exact2}
  The functor taking $M$ to $C^\bullet (G,M)$ preserves finite colimits.
\end{lemma}

\begin{lemma}
  \lean{groupCohomology.cochainsFunctor_Exact}
  The functor taking $M$ to $C^\bullet (G,M)$ is exact.
  I.e. if $0 \to A \to B \to C \to 0$ be a short exact sequence of $G$-modules.
  Then the corresponding sequence of cochain complexes is exact:
  \[
    0 \to C^n(G,A) \to C^n(G,B) \to C^n(G,C) \to 0.
  \]
\end{lemma}

\begin{proof}
  \uses{lem:groupCohomology.exact1, lem:groupCohomology.exact2}
  This follows automatically from lemmas \ref{lem:groupCohomology.exact1}
  and \ref{lem:groupCohomology.exact2} (by \texttt{exact?}).
\end{proof}


\begin{definition}[Long exact sequences]
  \lean{grougroupCohomology.δ}
  Given a short exact sequence $0 \to A \stackrel{f}\to B \stackrel{g}\to C \to 0$ of $G$-modules,
  the corresponding sequence of cochain complexes is exact:
  $0 \to C^n(G,A) \to C^n(G,B) \to C^n(G,C) \to 0$.
  This implies that there exist "connecting homomorphisms" $\delta : H^n(G,C) \to H^{n+1}(G,A)$,
  such that the following is a long exact sequence:
  \[
    0 \to H^0(G,A) \stackrel{f_*}\to H^0(G,B) \stackrel{g_*}\to H^0(G,C) \stackrel{\delta}\to
    H^1(G,A) \stackrel{f_*}\to H^1(G,B) \stackrel{g_*} \to H^1(G,C) \stackrel{\delta}\to \cdots.
  \]
\end{definition}



We shall regard $\Z$ as a $G$-module on which the action of $G$ is trivial.
By a \emph{projective resolution of $\Z$}, we shall mean an exact sequence of $G$-modules
\[
  \cdots \to P_2 \stackrel{d_1}\to P_1 \stackrel{d_0}\to P_0 \to \Z \to 0,
\]
in which each $P_i$ projective.
An example of such a resolution is the bar resolution,
which is constructed in Mathlib.
Given any projective resolution of $\Z$ and any $G$-module $M$, we may define a cochain
complex of additive groups $C^n(G,M) = Hom_{\Z[G]}(P_n,M)$. The coboundary maps
$\partial^{(n)} : C^n(G,M) \to C^{n+1}(G,M)$ are defined by $\partial^{(n)} f = f \circ d_n$.
Regardless of the resolution $P_\bullet$,
the cohomology groups $H^n(G,M)$ are isomorphic to the cohomology groups
of the cochain complex $C^\bullet(G,M)$. This isomorphism is already in Mathlib.

(say something about inflation and restriction).





\section{Cohomology of cyclic groups}

(transfer lean file on the resolution across)

Let $G$ be a finite cyclic group of order $n$. We shall write $\xi$ for a generator of $G$.
The element $N$ in the group ring $\Z[G]$ is defined to be the sum of the group elements.

There is a special projective resolution of $\Z$ in which all of the modules $P_n$ are $\Z[G]$:
\[
  \Z[G] \stackrel{1-\xi} \to \Z[G] \stackrel{N}\to \Z[G] \stackrel{1-\xi}
  \to \Z[G] \stackrel{N}\to \Z[G] \stackrel{1-\xi} \to \Z[G] \stackrel{\epsilon}\to \Z.
\]
Here the map $\epsilon : \Z[G] \to \Z$ is the \emph{augmentation map}, defined by
\[
  \epsilon \left(\sum_{g \in G} c_g \bullet g\right) = \sum_{g \in G} c_g.
\]
We shall call this resolution $P_n$, so the modules in the resulotion are all $P_n = \Z[G]$
and the maps $d_n : P_{n+1} \to P_n$ are given by
\[
  d_n(m) = \begin{cases}
    m \cdot (1-\xi) & \text{if $n$ is even,}\\
    m \cdot N & \text{if $n$ is odd.}
  \end{cases}
\]
Given a $\Z[G]$-module $M$, the cohomology groups of $M$ may be calculated by
taking the cohomology of the cochain complex $C^n(G,M) = Hom_{\Z[G]}(P_n,M)$.
We may identify each of the cochain groups $C^n(G,M)$ with $M$
(this is \texttt{Rep.leftRegularHomEquiv} in Mathlib).
With this identification, the coboundary maps $\partial^{(n)} : C^n(G,M) \to C^{n+1}(G,M)$
are given by
\[
  \partial^{(n)} (m) = \begin{cases}
    (1-\xi) \cdot m & \text{if $n$ is even,}\\
    N \cdot m & \text{if $n$ is odd.}
  \end{cases}
\]
Hence the cochcain complex $C^\bullet(G,M)$ looks like this:
\[
  0 \to M \stackrel{1-\xi}\to M \stackrel{N} \to M \to
    \stackrel{1-\xi}\to M \stackrel{N} \to M \to \cdots.
\]

\begin{lemma}
  The kernel of the map $\partial^{(even)}: M \to M$
  is $M^G$.
\end{lemma}

\begin{proof}
  An element $m$ is in the kernel iff $(1-\xi)m=0$.
  This is equivalent to $m=\xi m$. Since $\xi$ generates $G$, this is
  equivalent to $m$ being $G$-invariant.
\end{proof}

Hence for $n > 0$ we have
\[
  H^n(G,M) \cong \begin{cases}
    M^G / N\cdot M & \text{if $n$ is even,}\\
    M[N] / (1-\xi) \cdot M & \text{if $n$ is odd.}
  \end{cases}
\]

\begin{theorem}\label{cyclic group cohomology with values in Z}
  Let $\Z$ be the trivial $G$-module.
  Then $H^1(G,\Z)=0$ and $H^2(G,\Z) \cong \Z/n\Z$, where $n = |G|$.
\end{theorem}

\begin{proof}
  In the case that the module $M$ is $\Z$, the map $\partial^{(even)} : \Z \to \Z$ is zero
  and the map $\partial^{(odd)} : \Z\to \Z$ is multiplication by $n$.
  The result follows immediately from this.
\end{proof}


\section{Herbrand Quotents}

Let $A$ be an additive commutative group, and suppose we have two endomorphisms
$f,g : A \to A$ such that $f \circ g = g \circ f = 0$.
Then the kernel of $f$ contains the image of $g$ and the kernel of $g$ contains the image of $f$,
and we may define the Herbrand quotient $q_{f,g} \in \Q$ by
\[
  q_{f,g} = \frac{| \ker f / \image g|}{| \ker g / \image f|}.
\]
(If either the numerator or denominator is infinite then $q_{f,g}$ takes the default value $0$.)

An example of this situation is where $M$ is a module for a finite cyclic group $G$.
The maps $f,g : M \to M$ are defined to be $\partial^{(even)} = 1-\xi$ and $\partial^{(odd)} = N$
respectively.
In this case we write
\[
  h(G,M) = q_{1-\xi,N},
\]
and we have
\[
  h(G,M) = \frac{|H^2(G,M)|}{|H^1(G,M)|}.
\]

\begin{theorem}
  \label{Herbrand of short exact sequence}
  Suppose we have a commutative diagram with exact rows:
  \[
    \begin{matrix}
      0 &\to& A &\to& B &\to& C &\to& 0 \\
      &&\downarrow f_A &&\downarrow f_B &&\downarrow f_C \\
      0 &\to& A &\to& B &\to& C &\to& 0 \\
      &&\downarrow g_A &&\downarrow g_B &&\downarrow g_C \\
      0 &\to& A &\to& B &\to& C &\to& 0 \\
    \end{matrix}
  \]
  Suppose that the pairs $f_*,g_*$ satisfy compose to zero in either direction
  for $*=A,B,C$, and thus have Herbrand quotients.
  Assume that at least two of $q(A)$, $q(B)$ and $q(C)$ are non-zero.
  Then all are non-zero and we have
  \[
    q(B) = q(A) q(C).
  \]
  Here we are using the notation $q(*) = q_{f_*,g_*}$.
\end{theorem}


In the special case that $M$ is a module over a cyclic group, the theorem gives:

\begin{corollary}
  Suppose $G$ is a finite cyclic group and we have a short exact sequence of $G$-modules
  \[
    0 \to A \to B \to C \to 0.
  \]
  Then $h(G,B) = h(G,A) h(G,C)$.
\end{corollary}

\begin{theorem}
  \label{Herbrand of finite}
  If $A$ is finite then $q_{f,g} = 1$.
\end{theorem}

\begin{proof}
  By the first isomorphism theorem $|\image f| = \frac{|A|}{|\ker f|}$, and similarly with $g$.
  Substituting this into the definition we have
  \[
    q_{f,g} = \frac{|\ker f| / (|A| / |\ker g|)}{|\ker g| / (|A| / |\ker f|)}
    = 1.
  \]
\end{proof}

\begin{lemma}
  Let $G$ be a cyclic group of order $n$ and let $\Z$ be the trivial $G$-module.
  Then we have
  \[
    h(G,\Z) = n.
  \]
\end{lemma}

\begin{proof}
  We have calculated both cohomology groups in
  Theorem \ref{cyclic group cohomology with values in Z}.
\end{proof}


\section{Splitting modules}

In this section $G$ is a group and $M$ is a $G$-module.
We suppose that we have a 2-cocycle $\sigma \in Z^2(G,M)$.
From such a $\sigma$ we shall construct a $G$-module $S$, called the \emph{splitting module} of
$\sigma$.
We shall show that there is an exact sequence of $G$-modules
\[
  0 \to M \to S \to \Z[G] \to \Z \stackrel{\epsilon}\to 0.
\]
The image of $\sigma$ in $H^2(G,S)$ is $0$.

(fill in the definition, transfer lean file across.)


\section{Tate's Theorem}

If $M$ satisfies certain conditions, then Tate has proved that $H^n(G,S)=0$ for all $n>0$,
and more generally $H^n_{Tate}(G,S)=0$ for all $n \in \Z$.
If this is the case then we see that $M$ is a dimension-shift of $\Z$, so that we have
isomorphisms for all $n \in \Z$:
\[
  H^n_{Tate}(G,\Z) \cong H^{n+2}(G,M).
\]

(state and prove the theorem. Maybe prove it with weaker hypotheses in the case $n=-2$.)
