\chapter{Local Class Field Theory}

\section{Fundamental classes}
Let $G$ be a finite group and $M$ a representation of $G$ over $\Z$.
A generator $\sigma \in H^2(G,M)$ is called a fundamental class if for every subgroup $H \le G$,
\begin{itemize}
	\item
	$H^1(H,M) = 0$,
	\item
	$H^2(H,M)$ is cyclic of order $n$
	(in such circumstances the restriction of $\sigma$ is a generator).
\end{itemize}
If such a class exists, then we have a ``reciprocity isomorphism''
\[
	G^{ab} \cong M^G / N_G(M).
\]

The simplest example of a fundamental class is in the case
that $G$ is a cyclic group of order $n$ and $M$ is the trivial representation $\Z$.
It follows immediately that the element $\sigma_1 \in H^2(G,\Z)$ defined in
\ref{lem:H2 cyclic Z} is a fundamental class.
In this case the reciprocity isomorphism takes the form
\[
	G^{ab} \cong H^0_{Tate} (G,\Z) \cong \Z/n.
\]
(The isomorphism depends on $\sigma_1$, which depends on a choice of generator $g \in G$).

In this chapter we shall consider a slightly more complicated examples in which the group $G$ is
the Galois group of a finite extension $l/k$ of (non-archimedean) local fields, and the
representation $M$ is $l^\times$.
In this chapter we will describe the construction of the fundamental classes in this case.
By abstract class field theory, the fundamental classes give rise to a reciprocity isomorphism
\[
	\Gal(l/k)^{ab} \cong k^\times / N_{l/k} l^\times.
\]
The construction of this isomorphism and its properties are known as local class field theory.

We shall always use the notation $H^\bullet(l/k,-)$ as an abbreviation
for $H^\bullet(\Gal(l/k),-)$.
The following result (called Hilbert's theorem 90) is already in Mathlib.

\begin{theorem}\label{thm:hilbert 90}
	\mathlibok
	Let $l/k$ be a finite Galois extension of fields.
	Then $H^1(l/k, l^\times) \cong 0$.
\end{theorem}

\begin{theorem}\label{thm:additive field trivial}
	Let $l/k$ be a finite Galois extension of fields.
	Then there is an isomorphism of $\Gal(l/k)$-representations:
	\[
		l \cong \ind_1(k).
	\]
	In particular $l$ has trivial cohomology.
\end{theorem}

\begin{proof}
	Recall from Galois theory that there is a normal basis for $l$ over $k$, i.e. a basis of the
	form
	\[
		\{x \bullet b_0 : x \in \Gal(l/k)\}.
	\]
	Define a map $\ind_1(k) \cong l$ by
	\[
		\Phi (f : G \to k) = \sum_{x \in \Gal(l/k)} f(x) \cdot x \bullet b_0.
	\]
	This is clearly a linear bijection; we check that it commutes with the Galois action:
	\[
		\Phi(g\bullet f)
		= \sum_x f(g^{-1}x) \cdot x b_0
		= \sum_x f(x) \cdot (gx) \bullet b_0
		= g \bullet \Phi(f).
	\]
\end{proof}






\section{The Herbrand quotient of \texorpdfstring{$l^{\times}$}{l*}}

In this section we'll prove that for a cyclic extension $l/k$ of local fields,
$h(l/k,l^\times) = [l:k]$.

\begin{lemma}
	\label{lem:exists additive trivial}
	\uses{lem:induced' trivial Tate}
	Let $l/k$ be a Galois extension of local fields and
	let $U$ be any neighbourhood of $0$ in $l$.
	There is a Galois-invariant compact open subgroup $L \subseteq U$
	which has trivial Tate cohomology.
\end{lemma}

\begin{proof}
	Let $P$ be the maximal ideal of $\cO_l$.
	Choose $n$ such that $P^n \subseteq U$.
	Choose a normal basis for $l$ over $k$ contained in $P^n$
	and let $L$ be the span of that basis over $\cO_k$.
	We therefore have an isomorphism of Galois modules $L \cong \ind_1(\cO_l)$,
	and induced representations have trivial Tate cohomology.
\end{proof}

\begin{lemma} \label{lem:herbrand compact open additive}
	\uses{lem:exists additive trivial,
		lem:herbrand finite,
		lem:herbrand ses}
	Suppose $l/k$ is a cyclic extension.
	Let $M \subset l$ be a compact open subrepresentation.
	Then $h(l/k,M)=1$.
\end{lemma}

\begin{proof}
	Choose $L \subseteq M$ as in lemma \ref{lem:exists additive trivial}.
	Since $L$ has finite index in $M$ we have $h(M) = h(L) = 1$.
\end{proof}


\begin{lemma}\label{lem:local isomorphism}
	There is a non-zero ideal $P^n \subset \cO_l$ such that the
	exponential and logarithm maps give inverse isomorphisms
	\[
		(1 + P^n, \times) \cong (P^n,+).
	\]
	This isomorphism commutes with the action of the Galois group.
\end{lemma}

\begin{proof}
	Choose $n$ large enough so that for all $r > 0$,
	\[
		rn - v_l(r!) \ge n, \qquad
		rn - v_l(r!) \stackrel{r \to \infty}\to \infty.
	\]
	Then $\exp(x)$ converges for all $x \in P^n$ to an element of $1 + P^n$ and
	$\log(1+x)$ converges to an element of $P^n$.
	Hence both $\exp(\log(1+x))$ and $\log(exp(x))$ converge.
	The identities $\exp(\log(1+x))=1+x$ etc. follow by observing that they are true as
	an equations of power series
\end{proof}


\begin{lemma}\label{lem:herbrand local units}
	If $l/k$ is a cyclic extension, then $h(l/k, \cO_l^\times) = 1$.
\end{lemma}

\begin{proof}
	Choose a subgroup $1+P^n$ as in the provious lemma.
	Since $\cO_l^\times / (1+P^n)$ is finite we have
	\[
		h(\cO_l^\times) = h(1+P^n) = h(P^n).
	\]
	The right hand side is $1$ by \ref{lem:herbrand compact open additive}.
\end{proof}

\begin{lemma} \label{lem:herbrand Z}
	If $G$ is a cyclic group and $\Z$ has the trivial action of $G$ then $h(G,\Z) = |G|$.
\end{lemma}

\begin{proof}
	The map	$N_G : \Z \to \Z$ is multiplication by $G$.
	Therefore
	\[
		Z^0_{Tate} = \Z^G = \Z, \qquad
		B^0_Tate = \image(N_G : \Z \to \Z) = n\Z.
	\]
	This shows that $H^0_{Tate}(G,\Z)\cong \Z / n\Z$.
	Also,	$H^1_{Tate}(G,\Z) \cong Hom(G,\Z) \cong 0$.
\end{proof}

\begin{lemma} \label{lem:herbrand local l*}
	If $l/k$ is a cyclic extension then $h(l/k, l^\times)= [l:k]$.
\end{lemma}

\begin{proof}
	We have a short exact sequence of representations
	\[
		0 \to \cO_l^\times \to l^\times \to \Z \to 0,
 	\]
	where the second map is the valuation.
	We've shown in \ref{lem:herbrand local units}, \ref{lem:herbrand Z}
	that $h(l/k,\cO^\times)=1$ and $h(l/k,\Z) = [l:k]$.
	Therefore $h(l/k,l^\times) = [l:k]$.
\end{proof}


\begin{lemma} \label{lem:local H2 l*}
	If $l/k$ is a cyclic extension of local fields then $|H^2(l/k,l^\times)| = [l:k]$.
\end{lemma}




\section{An upper bound for \texorpdfstring{$H^2(l/k,l^\times)$}{$H^2(l/k,l*)$}}

\begin{theorem} \label{lem:local H2 upper bound}
	Let $l/k$ be a Galois extenion of local fields.
	Then $|H^2(l/k,l^\times)| \le [l:k]$.
\end{theorem}

\begin{proof}
	This is proved by induction on $k$, starting with $k=l$ and moving down in
	cyclic steps. Assume the result for $l/k$ and let $k_0$ be a subfield of $k$ with $k/k_0$ cyclic.
	We have an inflation-restriction sequence
	\[
		0 \to H^2(k/k_0, k^\times) \to H^2( l/k_0, l^\times) \to H^2(l/k, l^\times).
	\]
	The first term has order $[k : k_0]$ by \ref{lem:local H2 l*}, and the last term has
	order at most $[l:k]$ by the inductive hypothesis.
	Thereofore $H^2( l/k_0, l^\times)$ has order at most $[l:k] \times [k : k_0] = [l: k_0]$.

	Note that here we have used the fact that the Galois group of an extension of local fields is
	solvable.
	One can prove the same theorem without using this fact since the Sylow $p$ subgroup
	of $H^n(G,M)$ is isomorphic to a certain subgroup of $H^n(S_p,M)$ where $S_p$ is
	the Sylow $p$-subgroup of $G$ (which is solvable).
\end{proof}







\section{Fundamental classes in unramified extensions}

In this section we assume that $l/k$ is unramified.
In this case $\Gal(l/k)$ may be identified with $\Gal(\F_l / \F_k)$ where
$\F_l$ and $\F_k$ are the residue class fields of $l$ and $k$ respectively.
This group is cyclic and is generated by the Frobenius element $F_k$.
If we choose a uniformizer $\pi_k$ in $k$ (i.e. a generator for the maximal ideal in $\cO_k$)
then $\pi_k$ is also a uniformizer in $l$, so we may identify $\F_l$ with $\cO_l / \pi_k \cO_l$.

\begin{lemma} \label{lem:finite field trivial}
	The Galois modules $\F_l$ and $\F_l^\times$ have trivial cohomology.
\end{lemma}

\begin{proof}
	By periodicity, it's suffient to prove that $H^1$ and $H^2$ are trivial.
	Also, since $\F_l$ and $\F_l^\times$ are finite, they both have Herbrand quotient $1$,
	so it's enough to prove that $H^1$ is trivial.
	This follows from \ref{thm:hilbert 90} and \ref{thm:additive field trivial}.
\end{proof}


\begin{lemma} \label{lem:unramified additive trivial}
	If $l/k$ is unramified then there is a normal basis for $\cO_l$ over $\cO_k$.
	Hence there is an isomorphism of Galois representations $\cO_l \cong \ind_1 \cO_k$.
	In particular $\cO_l$ has trivial cohomology.
\end{lemma}

\begin{proof}
	By \ref{thm:additive field trivial}
	we may choose $x_0 \in \F_l$
	such that $\{\sigma \bullet x_0 :\sigma \in \Gal(l/k)\}$ is a normal basis.
	Let $y \in \cO_l$ be a lift of $x_0$.
	We claim that $\{\sigma \bullet y\}$ is a normal basis
	in $\cO_l$. It's sufficient to show that these vectors span $\cO_l$ over $\cO_k$.
	Choose any $z \in \cO_l$. By assumption we may solve the congruence
	\[
		z \equiv \sum \lambda_{\sigma,0} \sigma y \mod \pi_k.
		\qquad
		(\lambda_{\sigma,0} \in \cO_k).
	\]
	Similarly we may solve the congruence
	\[
		(z - \sum \lambda_{\sigma,0} \sigma y) / \pi_k \equiv \sum \lambda_{\sigma,1} \sigma y \mod \pi_k.
		\qquad
		(\lambda_{\sigma,1} \in \cO_k).
	\]
	This implies
	\[
		z \equiv \sum (\lambda_{\sigma,0} + \lambda_{\sigma_1} \pi_k) \sigma y \mod \pi_k^2.
	\]
	etc.
	Proceeding in this way, we construct convergent series
	$\lambda_{\sigma} = \sum \lambda_{\sigma,r}\pi^r \in \cO_k$,
	such that $z = \sum_\sigma \lambda_\sigma y$.
\end{proof}




\begin{lemma}	\label{lem:unramified units trivial}
	If $l/k$ is unramified then $\cO_l^\times$ has trivial cohomology.
\end{lemma}

\begin{proof}
	Recall (\ref{lem:local isomorphism}) that for $n$ sufficiently large we have isomorphisms of
	Galois modules:
	\[
		1 + P^n \cong P^n \cong \cO_l,
	\]
	where the first map is the logarithm and the second map is multiplication by $\pi_k^{-n}$.
	Hence by \ref{lem:unramified additive trivial}, the multiplictive subgroup $1+P^n$ has trivial
	cohomology.
	The long exact sequence now gives isomorphisms
	\[
		H^r(l/k,\cO_l^\times) \cong H^r(l/k,\cO_l^\times / (1+P^n)).
	\]
	We'll prove by induction on $n$ that $\cO_l^\times / (1+P^n)$ has trivial cohomology.
	In the case $n = 1$ we have
	\[
		\cO_l^\times / (1+P) \cong \F_l^\times.
	\]
	In this case the result follows from \ref{lem:finite field trivial}.

	For the inductive step we note that there is a short exact sequence of Galois modules
	\[
		0 \to \F_l \to  \cO_l^\times / (1+P^{n+1})  \to\cO_l^\times / (1+P^n)  \to  0,
	\]
	where we have identified $P^n / P^{n+1}$ with $\F_l$.
	By the inductive hypothesis, we assume that $\cO_l^\times / (1+P^n)$ has tivial cohomology.
	By \ref{lem:finite field trivial} $\F_l$ has trivial cohomology.
	Hence by the long exact seqeunce, $\cO_l^\times / (1+P^{n+1})$ has trivial cohomology.
\end{proof}


\begin{corollary}
	Let $l/k$ be an unramified extension of local fields.
	Then there are isomorphisms
	\[
		H^\bullet_{Tate}(l/k,l^\times) \cong H^\bullet_{Tate} (l/k,\Z)
	\]
	defined by the valuation map $v : l^\times \to \Z$.
	The inverse map is defined by $n \mapsto \pi_k^n$, and does not depend on the choice of $\pi_k$.
\end{corollary}

\begin{proof}
	This follows from the long exact sequence using \ref{lem:unramified units trivial}.
\end{proof}


\begin{lemma} \label{lem:unrammified fundamental class}
	Let $l/k$ be an unramified cyclic extension of local fields.
	Then $H^2(l/k,l^\times)$ is cyclic of order $[l:k]$.
	It is generated by the cohomology class of the following cocycle
	\[
		\sigma_{l/k} (F_k^r, F_k^s) =
		\begin{cases}
			1 & r + s < [l:k], \\
			\pi_k & r + s \ge [l:k].
		\end{cases}
	\]
	Here $F_k$ is the Frobenius element generating $\Gal(l/k)$ and $r$ and $s$
	are chosen to be integers in the range $0 \le r,s <[l:k]$.
\end{lemma}

\begin{proof}
	This follows from the previous result together with
	the description of $H^2(l/k,\Z)$.
\end{proof}


\begin{lemma}
	Let $m / l / k$ be an unramified tower of extensions of local fields
	Then the restriction to $m/l$ of $\sigma_{m/k}$ is $\sigma_{m/l}$.
	I.e. $\sigma$ is a fundamental class.
\end{lemma}

\begin{proof}
	Up to cohomology, $\sigma_{l/k}$ does not depent on the choice of
	uniformizer, so we may assume $\pi_k=\pi_l$
	in our definitions of $\sigma_{m/k}$ and $\sigma_{m/l}$.
	We have $F_l = F_k^f$ where $f = [l:k]$.
	Hence
	\begin{align*}
		\sigma_{m/k}(F_l^r, F_l^s)
		&=
		\sigma_{m/k}(F_k^{fr}, F_k^{fs})\\
		&=
		\begin{cases}
			1 & fr + rs < [m:k] \\
			\pi_k & fr + fs \ge [m:k]
		\end{cases}\\
		&=
		\begin{cases}
			1 & r + s < [m:l] \\
			\pi_k & r + s \ge [m:l]
		\end{cases}\\
		&=\sigma_{m/l}(F_l^r,F_l^s).
	\end{align*}
\end{proof}


By \ref{lem:H2 cyclic Z} we have an isomorphim $H^2(l/k, l^\times) \cong \Z / n \Z$ defined by
\[
	\inv_{l/k}(\sigma) = \sum_{i=1}^{[l:k]} v_k(\sigma(F_k^i, F_k)).
\]
The class $\sigma_{l/k}$ maps to $1 \in \Z/n\Z$.







\section{Construction of fundamental classes}

Now let $l/k$ be a Galois extension of local fields of degree $n$ and let $l'$ be
the unramified extension of the same degree. Le shal let $m$ be the field generated by $l$ and $l'$.
Let $e$ and $f$ be the ramification index and inertia degree of $l/k$.
Then we have $[m:l] = e$ and $v_l(\pi_k)= e$ and
\[
	F_l|l' = F_k^f.
\]
We have a class $\sigma_{l'/k} \in H^2(l'/k,l'^\times)$, and by inflation we can regard
$\sigma_{l'/k}$ as an element of $H^2(m/k,m^\times)$.
We also have inflation restriction sequences
\[
	0 \to H^2(l/k, l^\times) \to H^2(m/k , m^\times) \to H^2(m/l, m^\times)
\]
We'll calculate the image of $\sigma_{l'/k}$ in $H^2(m/l, m^\times)$.
Since $m/l$ is unramified, we have an isomorphism
\[
	\inv_{m/l} : H^2(m/l, m^\times) \cong \Z/ e.
\]
We have
\[
	\inv_{m/l}(\sigma_{l'/k})
	=
	\sum_{i=0}^{e-1} v_l (\sigma_{l'/k}(F_l^i,F_l))
	=
	\sum_{i=0}^{e-1} v_l (\sigma_{l'/k}(F_k^{if},F_k^f))
	=
	v_l (\pi_k)
	= e
	\equiv 0 \bmod [m:l].
\]
Therefore, there is a unique preimage of $\inf_m(\sigma_{l'/k})$ in $H^2(l/k,l^\times)$.
We shall call this preimage $\sigma_{l/k}$.

\begin{theorem} \label{H2 local cyclic}
	For every finite Galois extension $l/k$ of local fields,
	$H^2(l/k,l^\times)$ is a cyclic group of order $[l:k]$ generated by $\sigma_{l/k}$.
\end{theorem}

\begin{proof}
	Since the inflation maps are injective in this context, $\sigma_{l/k}$ has order $[l:k]$.
	We've already proved that the cohomology group has no more than $[l:k]$ elements, so
	it is generated by $\sigma_{l/k}$.
\end{proof}
