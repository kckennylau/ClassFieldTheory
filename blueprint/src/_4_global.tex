\chapter{Global Class Field Theory}

In this chapter we let $l/k$ be a finite Galois extension of
algebraic number fields. We shall consider the idele class group $\Cl_l = \A_l^\times / l^\times$
as a module for the Galois group $\Gal(l/k)$ and we shall describe the construction of
fundamental classes in $H^2(l/k, \Cl_l)$.
These classes give rise to a reciprocity isomorphism
\[
	\Gal(l/k)^{\ab} \cong \Cl_k / N(\Cl_l).
\]
We would therefore like to prove for all intermediate fields $l / m / k$:
\begin{itemize}
	\item
	$H^1(l/m, \Cl_l) = 0$,
	\item
	$H^2(l/m, \Cl_l)$ is cyclic of order $[l:m]$.
\end{itemize}

We note the following consequence of Hilbert's theorem 90:

\begin{lemma} \label{lem:idele class invariants}
	\uses{thm:hilbert 90}
	Let $l/k$ be any Galois extension of number fields.
	The map $\Cl_k \to \Cl_l ^ {\Gal(l/k)}$ is an isomorphism.
\end{lemma}

\begin{proof}
	We have a short exact sequence
	\[
		0 \to l^\times \to \A_l^\times \to \Cl_l \to 0.
	\]
	Taking $\Gal(l/k)$-invariants gives a long exact sequence beginning with
	\[
		0 \to k^\times \to \A_k^\times \to \Cl_l^{\Gal(l/k)} \to H^1(l/k, l^\times).
	\]
	The last term is zero by \ref{thm:hilbert 90}. This implies the result.
\end{proof}

By the lemma, we may regard $H^0_{\Tate}(l/k,\Cl_l)$ as the quotient
$\Cl_k / N_{l/k} \Cl_l$.
Furthmore, the inflation map for a tower of Galois extensions $l/m/k$ takes the form
\[
	H^\bullet(m/k, \Cl_m) \to H^\bullet(l/k, \Cl_l).
\]







\section{Choice of $S$}

Let $S$ be a finite set of places of $k$, containing all of the infinite primes and all
of the primes which ramify in $l$.
We shall use the notation
\[
	\A_{l,S} = \prod_{v \in S} \prod_{w | v} l_w \times \prod_{v \in S} \prod_{w | v} \cO_v.
\]
We also write $\cO_{l,S}$ for the ring $S$-integers in $l$:
\[
	\cO_{l,S} = \{x \in l, \forall v \not\in S, \forall w | v, |x|_v \le 1\} = k \cap \A_{l,S}.
\]
We regard $\cO_{l,S}$ as a subring of $\A_{l,S}$ and
$\cO_{l,S}^\times$ as a subgroup of $\A_{l,S}^\times$.

Note that the quotient $\A_l^\times / \A_{l,S}^\times$ is naturally isomorphic to the
group of fractional ideals of $\cO_{l,S}$.
By adding more primes to $S$ if necessary, we may assume that $\cO_{k,S}$ and $\cO_{l,S}$ are both
principal ideal domains (that can be achieved by adding to $S$ the primes $P$ whose ideal classes
generate the class groups of $k$ and $l$).
This implies
\[
	\A_{l}^\times = \A_{l,S}^\times l^\times,
	\qquad
	\A_k^\times = \A_{k,S}^\times k^\times,
\]
and therefore
\[
	\Cl_{l} = \A_{l,S}^\times / \cO_{l,S}^\times, \qquad
	\Cl_{k} = \A_{k,S}^\times / \cO_{k,S}^\times.
\]
The big advantage of working with $\A_{l,S}^\times$ and $\cO_{l,S}$ instead of $\A_l^\times$ and
$l^\times$ is that $\A_{l,S}^\times$ and $\cO_{l,S}$ have well-defined Herbrand quotients,
whereas the cohomology groups of $\A_l^\times$ and $l^\times$ are infinite.





\section{The Herbrand quotient of the $S$-ideles}

Let $v$ be a place of $k$ and $\hat v$ a place of $l$ above $v$.
We'll write $D_{\hat v}$ the decomposition group at $\hat v$.

\begin{lemma}
	Then there are isomorphisms
	\[
		\prod_{w | v} l_w^\times
		\cong
		\coind_{D_{\hat v}}^{\Gal(l/k)} l_{\hat v}^\times,
		\qquad
		\prod_{w | v} \cO_w^\times
		\cong
		\coind_{D_{\hat v}}^{\Gal(l/k)} \cO_{\hat v}^\times.
	\]
\end{lemma}

\begin{proof}
	Define a map $\Phi : \prod_{w | v} l_w^\times \to (\Gal(l/k) \to l_{\hat v}^\times)$ by
	\[
		\Phi ( x_w)
		=
		(g \mapsto g \bullet x_{g^{-1} \hat v}).
	\]
	Note that for $h \in D_{\hat v}$ we have
	\[
		\Phi (x_w) (hg) = h \bullet (g  x_{g^{-1} \hat v}) = h \bullet \Phi(x_w) (g).
	\]
	Therefore $\Phi(x_w)$ is actually in the subspace $\coind_{D_v} l_{\hat v}$, and it's
	easy to check that $\Phi$ gives a group isomorphism
	$\prod_{w | v} l_w \cong \coind_{D_v} l_{\hat v}$.
	We'll check that this map intertwines the actions of $G$:
	Note that for $g \in G$, then element $g \bullet (x_w)$ has $w$-coordinate
	$g \bullet x_{g^{-1}\bullet w}$.
	This implies
	\[
		\Phi( g \bullet (x_w))(h)
		=	h\bullet (g \bullet x)_{h^{-1} \bullet \hat v}
		=	h g \bullet x_{(hg)^{-1} \bullet \hat v}
		= \Phi (x_w) (hg).
	\]
	The proof of the other isomorphism is similar.
\end{proof}

\begin{lemma}
	There are isomorphisms for all $n > 0$
	\[
		H^n(l/k, \A_{S,l}^\times)
		\cong
		\prod_{v \in S} H^n(l_{\hat v} / k_v, l_{\hat v}^\times).
	\]
\end{lemma}

\begin{proof}
	We note that by the previous lemma we have
	\[
		\A_{S,l}^\times \cong \prod_{v \in S} \coind_{D_{\hat v}} l_{\hat v}^\times
		\times
		\prod_{v \not\in S} \coind_{D_{\hat v}} \cO_{\hat v}^\times
	\]
	By Shapiro's lemma we have
	\[
		H^n(l/k,\A_{S,l}^\times) \cong
		\prod_{v \in S} H^n(l_{\hat v}/k_v , l_{\hat v}^\times)
		\times
		\prod_{v \not\in S} H^n(l_{\hat v}/k_v , \cO_{\hat v}^\times)
	\]
	For $v \not\in S$, the extension $l_{\hat{v}}/l_v$ is unramified, and
	we have proved in such cases that $\cO_{\hat v}^\times$ has trivial cohomology.
\end{proof}


\begin{lemma}
	If $l/k$ is a cyclic extension then we have
	\[
		h(l/k, \A_{S,l} ^\times )
		=
		\prod_{v \in S} |D_{\hat v}|.
	\]
\end{lemma}

\begin{proof}
	This follows from the previous lemma, and the
	calculation of Herbrand quotients for local fields.
\end{proof}



\section{The Herbrand quotient of the $S$-units}

Define the \emph{logarithmic space} $V_S$ to be the following finite dimensional vector space
over the real numbers:
\[
	V_S = \prod_{v \in S} \prod_{w | v} \R.
\]
We consider $L_S$ as a representation of $\Gal(l/k)$, where the Galois action
permutes the places $w$ lying above each $v \in S$.
As a Galois representation we have
\[
	V_S \cong \prod_{v \in S} \ind_{D_{\hat v}}^{\Gal(l/k)} \R.
\]
Contained in $V_S$ we have a lattice $L_S$ consisting of vectors whose components are all in $\Z$.
Here we are using the word ``lattice'' to mean the $\Z$-space of a basis for $V_S$.
We have an isomorphism
\[
	L_S \cong \prod_{v \in S} \ind_{D_{\hat v}}^{\Gal(l/k)} \Z.
\]

\begin{lemma}
	If $l/k$ is a cyclic extension then
	$h(l/k,L_S) = \prod_{v \in S} |D_{\hat v}|$.
\end{lemma}

\begin{proof}
	This follows from Shapiro's lemma together with
	the calculation of the cohomology of a cyclic group with values in $\Z$.
\end{proof}

\begin{lemma}
	Let $l/k$ be cyclic and let $M$ be any Galois-invariant lattice in $V_S$.
	Then $h(l/k,M) = \prod_{v \in S} |D_{\hat v}|$
\end{lemma}

\begin{proof}
	The representations $M \otimes \Q$ and $L_S \otimes \Q$ have the same character
	(this is just the character of the representation $V_S$).
	Therefore the representations $M \otimes \Q$ and $L_S \otimes \Q$ are isomorphic.
	Hence $M$ is isomorphic to subrepresentation of finite index in $L_S$,
	so thay have the same Herbrand quotient.
\end{proof}


The vector $(1,1,\ldots,1) \in V_S$ in fixed by all elements of $\Gal(l/k)$, so it spans
a subrepresentation isomorphic to the trivial representation $\Z$.
Recall the we have a logarithmic map
\[
	\log_S : \cO_S^\times \to V_S,
\]
where the $w$-component of $\log_S(x)$ is $\log |x|_w$.
The kernel of $\log_S$ is the finite group of roots of unity in $k$.

\begin{theorem} \label{thm:Dirichlet unit theorem}
	$\log_S(\cO_S^\times)$ has zero intersection with $\Span (1,1,\ldots,1)$.
	The direct sum of these subrepresentations is a lattice in $V_S$.
\end{theorem}

\begin{corollary}
	Let $l/k$ be a cyclic extension. Then
	\[
		h(l/k,\cO_{l,S}^\times) = \frac{\prod_{v\in S} |D_{\hat v}|}{[l:k]}.
	\]
\end{corollary}

\begin{proof}
	Since $\log_S$ has finite kernel, the Herbrand quotient of $\cO_{l,S}^\times$ is
	equal to that of $\log_S(\cO_S^\times)$.
	By Dirichlet's unit theorem, $\log_S(\cO_S^\times) \oplus \Z$ is a lattice in
	$V_S$.
	We know the Herbrand quotient of $\log_S(\cO_S^\times) \oplus \Z$ from the calculation above,
	and the Herbrand quotient of $\Z$ is $[l:k]$.
\end{proof}

\begin{corollary}
	If $l/k$ is cyclic then $h(\A_{l,S}^\times / \cO_{l,S}^\times) = [l:k]$.
\end{corollary}




\section{Dirichlet Density}

\begin{definition} \label{def:Dirichlet density}
	Let $M$ be a set of primes of $\cO_k$.
	We'll say that $M$ has a \emph{Dirichlet density} $c \in \R$ if
	\[
		\sum_{P \in M} N(P)^{-s} \stackrel{s \to 1+}\sim c  \cdot \log\left(\frac{1}{s-1}\right)
	\]
	where $s$ tends to $1$ through the real numbers $s>1$.
\end{definition}


\begin{lemma}
	Suppose $M_1$ and $M_2$ are disjoint sets of primes of $\cO_l$.
	If two of the sets $M_1, M_2, M_1 \cup M_2$ have a Dirichlet density, then so does the third
	and we have
	\[
		\density(M_1 \cup M_2) = \density(M_1) + \density(M_2).
	\]
\end{lemma}

\begin{proof}
	This is trivial.
\end{proof}

\begin{lemma} \label{lem:Dirichlet density top}
	The set of all primes of $\cO_k$ has Dirichlet density $1$.
\end{lemma}

\begin{proof}
	Let $P$ be a prime. We have
	\[
		\left|N(P)^{-s} - \log\left( \frac{1}{1-N(P)^{-s}}\right)\right| \ll N(P)^{-2s}
	\]
	Since $\sum N(P)^{-2s}$ is bounded in the region $s > 1$ (this follows from the convergence of the
	Dedekind zeta function), we have
	\[
		\sum_P N(P)^{-s}
		\sim
		\sum_P \log\left( \frac{1}{1-N(P)^{-s}} \right)
		=
		\log \zeta_l(s),
	\]
	where $\zeta_l$ is the Dedekind zeta function (\texttt{NumberField.dedekindZeta}).
	By the analytic class number formula (\verb!NumberField.tendsto_sub_one_mul_dedekindZeta_nhdsGT!)
	there is a positive real number $r$ such that
	\[
		\zeta_l(s) = \frac{r}{s-1} + O(1) \qquad (s > 1).
	\]
	This implies
	\[
		\log(\zeta_l(s)) \sim \log\left( \frac{1}{s-1}\right).
	\]
\end{proof}



\begin{lemma} \label{lem:Dirichlet density degree one}
	The set of primes of $\cO_l$ of degree one has Dirichlet density $1$.
\end{lemma}

\begin{proof}
	Let $M$ be the set of primes of degree larger than one. It's sufficient to
	prove that $M$ has Dirichlet density $0$.
	We have
	\[
		\sum_{P \in M} N(P)^{-s}
		= \sum_{n \in \N} A(n) n^{-s},
	\]
	Where $A$ is the number of primes of degree $>1$ with norm $n$.
	Note that $A(n) \le [l:\Q]$. Also $A(n)=0$ unless $n=m^r$ for positive integer $m$ and
	$1 \le r \le [l:Q]$.
	This implies
	\[
		\sum_{P \in M} N(P)^{-s}
		\le [l:\Q] \sum_{r=2}^{[l:\Q]} \sum_{m =1} ^\infty m^{-rs}
		\le [l:\Q] (\zeta_\Q(2) + \cdots + \zeta_\Q([l:\Q]) ).
	\]
	Since the sum above is bounded, the set $M$ has Dirichlet density $0$.
\end{proof}


\begin{lemma} \label{lem:Dirichlet density split}
	Let $l/k$ be a finite Galois extension of number fields.
	Then the set of degree $1$ primes of $k$ which split completely in $l$
	has density $\frac{1}{[l:k]}$.
\end{lemma}

\begin{proof}
	Let $M_k$ be the set of degree $1$ primes of $k$ and $M_l$ the set of degree $1$ primes of $l$.
	Every prime $Q$ in $M_l$ lies above some prime $P \in M_k$.
	If there is a prime $Q$ above $P$, then there are precisely $[l:k]$ of them; this happens when $P$
	splits completely in $l$.

	Let $M$ be the set of $P \in M_k$ which split completely in $l$.
	Then we have
	\[
		\sum_{P \in M} N(P)^{-s}
		=
		\frac{1}{[l:k]} \sum_{Q \in M_l} N(Q)^{-s}
		\sim \frac{1}{[l:k]} \log\left( \frac{1}{s-1}\right).
	\]
	Here we have used \ref{lem:Dirichlet density degree one}.
\end{proof}





\section{Some $L$-functions}

\begin{lemma}
	$H^0_{\Tate}(l/k,\Cl_l)$ is finite.
\end{lemma}

\begin{proof}
	We have
	\begin{align*}
		\Cl_{l} / N(\Cl_l)
		& \cong \A_{k,S}^\times / \cO_{k,S}^\times N(\A_{l,S}^\times) \\
		& \cong \left(\prod_{v \in S} k_v^\times / N(l_{\hat v}^\times)\right) / \cO_{k,S}^\times,
	\end{align*}
	where $\hat v$ is a place of $l$ lying above $v$.
	The result follows because each of the groups $k_v^\times / N(l_w)$ is finite
	(in fact by the local reciprocity isomorphism this is isomorphic to the abelianization of the
	decomposition group of $\hat v$).
\end{proof}

Given any maximal ideal $P$ of $\cO_S$, we let $\pi_P$ be an idele whose
$P$-component is a uniformizer in $k_P$, and whose other components are all $1$.
The coset of $\pi_P$ in $H^0(l/k,\Cl_l)$ does not depend on the choice of uniformizer since
all the local units at $P$ are local norms from $l_{Q}$ for any $Q|P$
(because $P$ is unramified in $l$).
Equivalently, if we choose a generator $P=(\pi)$ then the coset of $\pi_P$ is the inverse of
the image of $\pi$ in $\prod_{v \in S} k_v^\times / N(l_{\hat v}^\times)$.
This clearly extends to a map
\[
	\iota : \textrm{non-zero fractional ideals of $\cO_S$}
	\to
	H^0_{\Tate}(l/k,\Cl_l),
\]
whose kernel is the sub group of ideals with a generator in
$\prod_{v \in S} k_v^\times / N(l_{\hat v}^\times)$.

\begin{definition}
	For any character $\chi : H^0_{\Tate}(l/k,\Cl_l) \to C^\times$ we define the $L$-function
	\[
		L(s,\chi)
		=
		\sum_{I} \chi(I) \cdot N(I)^{-s}
		=
		\prod_{P} \frac{1}{1-\chi(P) \cdot N(P)^{-s}}.
	\]
	Here $s$ is a complex number with real part greater than $1$; bith the product and the series
	converge absolutely in that region.
	In the sum, $I$ ranges over the non-zero ideals of $\cO_S$, and in the product $P$
	ranges over the maximal ideals of $\cO_S$.

	If $\chi$ is the trivial character, then $L(s,\chi)$ is
	(up to finitely many Euler factors for primes in $S$)
	equal to the Dedekind zera function of $k$.
\end{definition}


It's known that $L(s,\chi)$ has a meromorphic continuation to $\C$
(see for example Tate's thesis, which is chapter XV of \cite{cassells frohlich}).
We won't need such a strong result here; we can make do with the following:

\begin{lemma}[Weak lemma]
	If $\chi$ is a non-trivial character then $L(s,\chi)$ is
	bounded on the interval $(1,2)$.
\end{lemma}



\begin{lemma}
	Let $M$ be a set of primes of $\cO_S$ whose image in $H^0_{\Tate}(l/k,\Cl_l)$ is zero.
	There exists a real number $c$ depending only on $k$ and $l$,
	such that for all $s > 0$ we have:
	\[
		\sum_{p \in M} N(P)^{-s}
		\le \frac{1}{|H^0_{\Tate}(l/k,\Cl_l)|} \log\left(\frac{1}{s-1}\right) + c.
	\]
\end{lemma}


\begin{proof}
	Let $s > 1$. All the series in the following calculation converge absolutely in this region.
	The implied constants in the $O(1)$ terms do not depend on $s$.
	\begin{align*}
		\sum_{P \in M} |N(P)|^{-s}
		&= \frac{1}{|H^0_{\Tate}(l/k,\Cl_l)|} \sum_P \sum_\chi \chi(P) N(P)^{-s}\\
		&= \frac{1}{|H^0_{\Tate}(l/k,\Cl_l)|}
		\sum_P \sum_\chi -\log(1-\chi(P) N(P)^{-s}) + O(1)\\
		&= \frac{1}{|H^0_{\Tate}(l/k,\Cl_l)|}
		\sum_P \sum_\chi -\log |1-\chi(P) N(P)^{-s}| + O(1)\\
		&= \frac{1}{|H^0_{\Tate}(l/k,\Cl_l)|}
		\sum_\chi \sum_P -\log |1-\chi(P) N(P)^{-s}| + O(1)\\
		&= \frac{1}{|H^0_{\Tate}(l/k,\Cl_l)|} \log \left(\prod_\chi |L(s,\chi)| \right) + O(1)\\
		&\le \frac{1}{|H^0_{\Tate}(l/k,\Cl_l)|} \log \left(\frac{1}{s-1}\right) + O(1).
	\end{align*}
	Interchanging the order of summation is justified because the series converge absolutely.
	We need to be slightly careful about which branch of the logarithm we are using here.
	In the expression $\log(1-\chi(P) N(P)^{-s})$ we shall mean the branch which is continuous
	on the ball of radius $1$, centred about $1$.
	The imaginary parts of $\log(1-\chi(P) N(P)^{-s})$ and $\log(1-\bar \chi(P) N(P)^{-s})$
	cancel out; this justifies replacing $\log(1-\chi(P) N(P)^{-s})$ by $\log |1-\chi(P) N(P)^{-s}|$.
\end{proof}

\begin{remark}
	In fact the density of the set $M$ in this lemma is precisely $\frac{1}{|H^0_{\Tate}(l/k,\Cl_l)|}$.
	This can be proved by showing that each $L(s,\chi)$ has a continuation to a neighbourhood
	of $s=1$, and is non-zero at $s=1$.
	However, proving this is more difficult than the weak lemma above, and we only need the inequality
	of the lemma.
\end{remark}






% \begin{lemma}
% 	There is a meromorphic continuation of $L(s,\chi)$
% 	to a neighbourhood of the interval $[\frac{1}{[l:k]},1]$.
% 	the region $\Re s > 1-\frac{1}{[k:\Q]}$.
% 	If $\chi=1$ then there is a simple pole at $s=1$ and no other poles in this region.
% 	If $\chi \ne 1$ then there are no poles.
% \end{lemma}


% We also need the following purely analytic theorem of Landau:

% \begin{theorem}
% 	Suppose we have a Dirichlet series $D(s) = \sum_{n=0}^\infty a_n \cdot n^{-s}$ with
% 	nonnegative real coefficients $a_n$.
% 	Suppose $D(s)$ has absolute convergence abscissa $\sigma$, i.e. it converges absolutely for $\Re s > \sigma$
% 	and not for $\Re s < \sigma$.
% 	Then there is no analytic continuation of $D(s)$ to any neighbourhood of $\sigma$.
% \end{theorem}

% \begin{proof}
% 	By rescaling the coefficients $a_n$ if necessary, we may assume that $\sigma=0$.
% 	Let's suppose that there does exist an nalytic continuation to some neighbourhood of $0$.
% 	This implies that the power series expansion of $D(s)$ about $s=1$ converges on a disk or radius
% 	greater than $1$. Let's assume that this power series converges absolutely one the closed
% 	disk of radius $1+\epsilon$.
% 	This power series expansion is:
% 	\[
% 		D(1+r)
% 		= \sum_{m=0}^\infty \frac{D^{(m)}(1)}{m!} r^m
% 		= \sum_{m=0}^\infty \left(\frac{\sum_{n=1}^\infty a_n (-m \log n )^m n^{-1}}{m!} r^m\right)
% 	\]
% 	Note that if $r < 0$ then every term in the double sum above is positive, so the order of
% 	summation may be swapped.
% 	Substituting $r = -1-\epsilon$ we get
% 	\[
% 		D(-\epsilon)
% 		=\sum_{n=1}^\infty a_n n^{-1}
% 		\left(\sum_{m=0}^\infty  \frac{(-m \log n )^m }{m!} (-1-\epsilon)^m\right).
% 	\]
% 	The inner series in the above expression is the power series expension of $n^{-\epsilon}$,
% 	which converges on the whole complex plane.
% 	We therefore have
% 	\[
% 		D(-\epsilon)
% 		=\sum_{n=1}^\infty a_n n^{-\epsilon}
% 	\]
% 	In particular the series on the right hand side converges,
% 	and the convergence is absolute because all the terms are non-negative real numbers.
% 	This implies absolute convergence of $D(s)$ in the half-place $\Re s > -\epsilon$, contradicting
% 	our assumption.
% \end{proof}


% \begin{theorem}
% 	For each non-trivial character $\chi$ we have $L(s,\chi) \ne 0$.
% \end{theorem}

% \begin{proof}
% 	Consider the following Dirichlet series:
% 	\[
% 		D(s) = \prod_\chi L(s,\chi).
% 	\]
% 	By examining the Euler factors for each prime, we can check that all the coefficients
% 	of $D(s)$ are nonnegative.
% 	Furthermore if $I$ is an ideal of $\cO_S$, then $D(s)$ has a term $N(I)^{-[l:k] s}$.
% 	This implies for real $s$ in the region of convergence
% 	\[
% 		D(s) \ge L(\chi_1,[l:k]s).
% 	\]
% 	If we assume that $L(1,\chi)=0$ for some $\chi$ then $D(s)$ is analytic
% 	in a neighbourhood of $[\frac{1}{[l:k]},1]$.
% 	Hence bu Landau's theorem, the Dirichlet seties for $D(s)$ converges absolutely on that region.
% 	It follows that $L(\chi_1, [l:k]s)$ is converges absolutely on that region.
% 	This contradicts the fact that $L(\chi_1, ds)$ has a pole at $\frac{1}{[l:k]}$
% \end{proof}





\section{The first inequality}

\begin{theorem} \label{thm:first inequality}
	\uses{lem:Dirichlet density split}
	For any finite Galois extension $l/k$ be have
	\[
		|H^0_{\Tate}(l/k, \Cl_{l}) | \le [l : k].
	\]
\end{theorem}

\begin{proof}
	The easiest proof of this theorem is rather like that of Dirichlet's primes in arithmetic
	progressions result.
	We'll sketch the argument.

	One can easily show that $H^0_{\Tate}(l/k, \Cl_{l})$ is finite.
	We have a function
	\[
		\textrm{primes in $\cO_{S,k}$} \to H^0_{\Tate}(l/k, \Cl_{l}),
	\]
	which takes a prime $P$ to the coset of an idele, whose $P$ component is
	a uniformizer in $k_P$, and whose other components are $1$.
	This coset does not depend on the choice of uniformizer because $P$ is unramified in $l$
	(by choice of $S$), and therefore every element of $\cO_P^\times$ is a local norm in the extension.

	By the method of Dirichlet's primes in arithmetic progressions theorem, we may show that
	the images in $H^0_{\Tate}(l/k, \Cl_{l})$ of the primes of $\cO_{k,S}$ are equidistributed,
	meaning that the set of primes in the preimage of an element of $H^0_{\Tate}(l/k, \Cl_{l})$ has
	Dirichlet density $\frac{1}{|H^0_{\Tate}(l/k, \Cl_{l})|}$. The proof of this equidistribution
	result can be reduced to proving $L(1,\chi) \ne 0$ for each non-trivial character
	$\chi$ of $H^0_{\Tate}(l/k,\Cl_l)$ (see for example chapter VIII of \cite{cassells frohlich}).

	Let $P$ be a degree one prime which splits completely in $l$.
	Then the norm map $(l \otimes k_P)^\times \to k_P$ is surjective,
	and therefore the image of $P$ in $H^0_{\Tate}(l/k, \Cl_{l})$ is $0$.
	We have shown in \ref{lem:Dirichlet density split} that the set of such primes has
	Dirichlet density $\frac{1}{[l:k]}$. It follows that
	\[
		\frac{1}{[l:k]} \le \frac{1}{|H^0_{\Tate}(l/k, \Cl_{l})|}.
	\]
	This proves the result.
\end{proof}


\begin{corollary} \label{cor:H1 H2 cyclic idele class}
	If $l/k$ is cyclic then $|H^2(l/k, \Cl_{l})| = [l:k]$ and $H^1(l/k, \Cl_{l}) = 0$.
\end{corollary}

\begin{proof}
	This follows immediately from (a) the first inequality, (b) the periodicity of
	the cohomology for a cyclic group, and (c) the calculation of the Herbrand quotient
	of $\Cl_{l}$.
\end{proof}

\begin{theorem}
	If $l/k$ is any finite Galois extension then $H^1(l/k, \Cl_{l}) = 0$
	and $|H^2(l/k, \Cl_{l})| \le [l:k]$.
\end{theorem}


\begin{proof}
	For each prime number $p$ dividing $[l:k]$ we let $k_p$ be the fixed
	field of a Sylow $p$-subgroup $S_p$ of $\Gal(l/k)$.
	By \ref{cor:cohomology sub Sylow}, it's suffient to prove
	\[
		H^1(l/k_p, \Cl_{l}) = 0, \qquad
		|H^2(l/k_p, \Cl_{l})| \le [l:k_p].
	\]
	Since $S_p$ is solvable, this reduces us to the case that $\Gal(l/k)$ is solvable.
	We'll prove the result by induction on $k$ starting with $k=l$ and working downwards
	in cyclic quotients.

	Clearly the result holds for $k=l$.
	Assume the result for a subfield $m$ of $l$ and let $m/k$ by cyclic.
	We have an inflation restriction sequence:
	\[
		0 \to H^1(m/k, \Cl_{m}) \to H^1(l/k, \Cl_{l}) \to H^1(l/m,\Cl_{l}).
	\]
	The first term is zero by \ref{cor:H1 H2 cyclic idele class} and the last term is zero by
	the inductive hypothesis.
	Therefore $H^1(l/k,\Cl_l)=0$.

	This implies that we have an inflation-restriction sequence
	\[
		0 \to H^2(m/k, \Cl_m) \to H^2(l/k, \Cl_l) \to H^2(l/m,\Cl_l).
	\]
	By the inductive hypothesis we have $|H^2(l/m,\Cl_l)| \le [l:m]$
	and by \ref{cor:H1 H2 cyclic idele class} we have $|H^2(m/k, \Cl_m)| \le [m : k]$.
	It follows that $|H^2(l/k,\Cl_l)| \le [l : k]$.
\end{proof}


To complete the construction of fundamental classes and the reciprocity isomorphism,
we need only show that there is an element in $H^2(l/k,\Cl_l)$ of order $[l:k]$.
Such an element is constructed first for a cyclic cyclotomic extension $l'/k$ with the same degree
as $l/k$.
It's then shown that the inflation of such a class to $ll'/k$ must split on $ll'/l$,
and must therefore be the inflation of an element of order $[l:k]$ in $H^2(l/k,\Cl_l)$.
